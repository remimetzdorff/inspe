\documentclass[12pt,a4paper]{article}
\usepackage[utf8]{inputenc}
\usepackage[french]{babel}
\usepackage[T1]{fontenc}
\usepackage{amsmath}
\usepackage{amsfonts}
\usepackage{amssymb}
\usepackage{graphicx}
\usepackage[left=2cm,right=2cm,top=2cm,bottom=2cm]{geometry}

\title{L'évaluation}
\author{Virginie Marchand}
\date{17/09/2020}

\begin{document}

\maketitle

Tester l'accès aux ressources informatiques par les élèves en envoyant uniquement sur pronote un travail simple à faire à la maison.

\section{Évaluations}

\subsection{Évaluation diagnostique}

Au début de la séquence avec comme objectif d'évaluer les prérequis.
Analyser les erreurs pour se rendre compte des conceptions.
Adapter l'apprentissage.

Attention à l'évaluation collective : elle ne permet pas vraiment d'avoir une image du niveau de la classe car seuls les bavards ou les forts parlent.
Elle n'est pas inutile mais il fauat faire attention aux objectifs derrière.

\subsection{Évaluation formative}

Savoir s'ils apprennent la leçon, s'il savent appliquer

\subsection{Évaluation sommative}

La seule obligatoire par l'institution.
On n'est pas obligé d'en faire une à chaque séquence.

\subsection{Évaluation certificative}

Les examens.

\end{document}