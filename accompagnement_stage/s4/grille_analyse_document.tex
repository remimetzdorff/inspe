\documentclass[12pt,a4paper]{article}
\usepackage[utf8]{inputenc}
\usepackage[french]{babel}
\usepackage[T1]{fontenc}
\usepackage{amsmath}
\usepackage{amsfonts}
\usepackage{amssymb}
\usepackage{graphicx}
\usepackage[left=2cm,right=2cm,top=2cm,bottom=2cm]{geometry}

\begin{document}

\thispagestyle{empty}
\begin{center}
\begin{tabular}{|l|c|c|c|c|}
\hline 
\textbf{Niveau de maitrise} & A & B & C & D \\ 
\hline 
\textbf{Forme} &  &  &  &  \\ 
\hline 
Police adaptée à tous &  &  &  &  \\ 
\hline 
Document aéré &  &  &  &  \\ 
\hline 
Document atractif : photo, image... &  &  &  &  \\ 
\hline 
Document structuré : numérotation, couleur, gras, surligné... &  &  &  &  \\ 
\hline 
Syntaxe, orthographe, expression &  &  &  &  \\ 
\hline 
Formulation des questions,  phrase : verbes en gras ou... &  &  &  &  \\ 
\hline 
Mise en évidence de l'essentiel &  &  &  &  \\ 
\hline 
Place prévue pour les écrits de l'élève : lignes tracées, renvoi au cahier... &  &  &  &  \\ 
\hline 
Si compétences, où les identifier : tableau récap, par question &  &  &  &  \\ 
\hline 
\textbf{Fond} &  &  &  &  \\ 
\hline 
Activité adaptée au programme, niveau &  &  &  &  \\ 
\hline 
Contextualisation &  &  &  &  \\ 
\hline 
Documents variés : texte, graphique, schéma &  &  &  &  \\ 
\hline 
Rigueur scientifique &  &  &  &  \\ 
\hline 
Objectifs de la séance identifiables &  &  &  &  \\ 
\hline 
Consignes bien formulées &  &  &  &  \\ 
\hline 
Choix démarches : mise en activité de l'élève &  &  &  &  \\ 
\hline 
Vocabulaire adapté aux élèves &  &  &  &  \\ 
\hline 
Compétences travaillées identifiées corectement &  &  &  &  \\ 
\hline 
\end{tabular} 
\end{center}

\end{document}