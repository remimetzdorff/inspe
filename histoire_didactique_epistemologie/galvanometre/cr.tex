\documentclass[fleqn,12pt,a4paper]{article}
\usepackage[utf8]{inputenc}
\usepackage[french]{babel}
\usepackage[T1]{fontenc}
\usepackage{amsmath}
\usepackage{amsfonts}
\usepackage{amssymb}
\usepackage{graphicx}
\usepackage[left=2cm,right=2cm,top=2cm,bottom=2cm]{geometry}
\usepackage[squaren,thinspace,thinqspace]{SIunits}							% écriture des nombres et unités

\usepackage{xcolor}																			% colors
\definecolor{gray_f}{RGB}{68,84,106}
\definecolor{gray_c}{RGB}{214,220,229}
\definecolor{gray_cc}{RGB}{245,245,245}
\definecolor{bleu_f}{RGB}{91,155,213}
\definecolor{bleu_c}{RGB}{222,235,247}
\definecolor{red_f}{RGB}{204,0,0}
\definecolor{red_c}{RGB}{245,204,204}
\definecolor{orange_f}{RGB}{237,125,49}
\definecolor{orange_c}{RGB}{251,229,214}
\definecolor{green_f}{RGB}{112,173,71}
\definecolor{green_c}{RGB}{226,240,217}
\definecolor{yellow_f}{RGB}{255,192,0}
\definecolor{yellow_c}{RGB}{255,242,204}

\usepackage[breaklinks]{hyperref}															% clickable links
\hypersetup{
    colorlinks=true,
    linkcolor=red_f,
    citecolor=bleu_f,
    filecolor=green_f,
    urlcolor=bleu_f
}

\begin{document}

\section*{24/02/2021}

\subsection*{Étude de la première bobinage résistif du voltmètre}

On s'intéresse à la bobinage résistif correspondant au premier calibre du voltmètre (calibre \unit{1{,}5}{V}, résistance affichée $R = 500$).

On décide de la dérouler.
Il y a sept tours de fil de cuivre tressé au-dessus d'un tour de scotch en toile puis un fil plus fin entouré de fil de soie.

On déroule la bobine en comptant le nombre de tours : il y a 680 tours.
La longueur totale du fil est mesurée avec un décamètre dans le couloir : le fil fin mesure \unit{21{,}8\pm 0{,}2}{m}.

On mesure la résistance linéique $R_l$ du fil :
\begin{itemize}
\item pour \unit{1}{m} : on mesure la résistance d'un mètre de fil avec un multimètre $ R = \unit{16{,}6 \pm 2}{\ohm}$.
\item pour \unit{10}{m} : on mesure la résistance de dix mètres de fil avec le multimètre $ R = \unit{177 \pm 5}{\ohm} $
\end{itemize}

La section du fil est mesurée avec un pied à coulisse puis avec un micromètre : on trouve \unit{200}{\micro\metre} à plusieurs endroits du fil (sans la soie).

Avec ces mesures on peut estimer la résistivité du matériau utilisé pour le fil $ R = \rho \frac{l}{\pi r^2} $ d'où
\[
\rho = R \frac{\pi r^2}{l}
\]
et on trouve $\rho \sim \unit{50 \times 10^{-8}}{\ohm\cdot\meter}$.
En prenant les valeurs de \href{https://webetab.ac-bordeaux.fr/Pedagogie/Physique/Physico/Electro/e07fil.htm}{ce site}, on peut penser que le fil est en constantan.

La résistance totale de la bobine est $21{,}8 \times 17{,}7 \approx \unit{386}{\ohm}$.



\end{document}