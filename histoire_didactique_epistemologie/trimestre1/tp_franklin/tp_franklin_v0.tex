\documentclass[12pt,a4paper]{article}
\usepackage{rmpackages}																% usual packages
\usepackage{rmtemplate}																% graphic charter
\usepackage{rmexocptce}																% for DS with cptce eval

%\cfoot{} 													% if no page number is needed
%\renewcommand\arraystretch{1.5}		% stretch table line height

\begin{document}

\section*{Prélude}

On verse un volume $V = \unit{300}{\liter}$ d'huile dans un bidon cylindrique de rayon $r = \unit{30}{cm}$.
\begin{enumerate}
\item Donner la valeur en $\mathrm{m^2}$ de la section $S$ du verre, c'est-à-dire la surface du disque de rayon $r$.
On rappelle la formule donnant la surface d'un disque en fonction de son rayon : $S = \pi \times r^2$.

\item Donner la valeur en mètre de la hauteur $h$ d'huile dans le verre.
On rappelle la formule donnant le volume $V$ d'un cylindre en fonction de sa section $S$ et de sa hauteur $h$ : $V = S \times h$.
\label{quest:volume}
\end{enumerate}

\section*{Mesurer une molécule d'huile}

Au $\text{XVIII}^\text{ème}$ siècle, Benjamin Franklin se promène au bord de l'étang de Clapham en Angleterre, et décide de verser un peu d'huile dans l'eau. Il observe alors qu'une tache se forme à la surface et s'étend rapidement jusqu'à couvrir presque un quart de la surface du plan d'eau.

Il faudra attendre 1890 pour que John W. Rayleigh reprenne cette expérience et en déduise la taille des molécules d'huile.

\begin{multicols}{2}
\center
\includegraphics[height=150pt]{images/franklin_lake.png}

L'étang agité un jour venteux.

\includegraphics[height=150pt]{images/franklin_lake_oiled.png}

La tache d'huile forme une étendue lisse.
\end{multicols}

\begin{doc}
\textbf{: extrait d'une lettre de Benjamin Franklin à la Royal Society (1774)}
\begin{multicols}{2}
\noindent
Texte original :

\textit{ \og
At length at Clapham where there is, on the common, a large pond, which I observed to be one day very rough with the wind, I fetched out a cruet of oil, and dropped a little of it on the water.
I saw it spread itself with surprising swiftness upon the surface.
The oil, though not more than a teaspoonful, produced an instant calm over a space several yards square, which spread amazingly and extended itself gradually until it reached the leeside, making all that quarter of the pond, perhaps half an acre, as smooth as a looking glass.
\fg{} }

\noindent
Traduction :

Enfin à Clapham où il y a, sur la commune, un grand étang que j'observai agité un jour de grand vent, je cherchai une burette d'huile et en laissai tomber un peu sur l'eau.
Je la vis se répandre sur la surface avec une rapidité surprenante.
L'huile, bien que moins d'une cuillère à café, produisit un calme immédiat sur une surface de plusieurs mètres carrés, qui se propagea incroyablement et s'étendit progressivement jusqu'à la côte rendant ce quart de l'étang, peut-être 2000\,m\textsuperscript{2}, aussi lisse qu'un miroir.
\end{multicols}
\end{doc}

\begin{enumerate}[resume]
\item Selon la lettre de Benjamin Franklin, quelle est la surface $S$ de la tache d'huile qu'il observe ?

\item Exprimer le volume d'une cuillère à café en litre puis en $\mathrm{m^3}$, sachant qu'à l'époque de Benjamin Franklin, les cuillères à café contenaient un volume de $\unit{2}{mL}$.

\item En vous aidant de la question~\ref{quest:volume}, déterminer la hauteur de la tache d'huile dans l'expérience de Benjamin Franklin.

\item On suppose que la tache d'huile n'est formée que d'une épaisseur de molécules d'huile.
Quelle est la taille d'une molécule d'huile ?
\end{enumerate}

\section*{Et un atome, c'est grand comment ?}

Une molécule d'huile est une grosse molécule composée de plusieurs dizaines d'atomes.
Un atome est environ dis fois plus petit qu'une telle molécule.
\begin{enumerate}[resume]
\item Quelle est la taille d'un atome ?
\end{enumerate}

\end{document}