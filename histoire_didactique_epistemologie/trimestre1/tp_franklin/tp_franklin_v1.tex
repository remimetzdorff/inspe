\documentclass[12pt,a4paper]{article}
\usepackage{rmpackages}																% usual packages
\usepackage{rmtemplate}																% graphic charter
\usepackage{rmexocptce}																% for DS with cptce eval

%\cfoot{} 													% if no page number is needed
%\renewcommand\arraystretch{1.5}		% stretch table line height

\begin{document}
\begin{header}
Mesurer une molécule
\end{header}

\section*{Objectif}

En vous appuyant sur l'expérience historique de Benjamin Franklin décrite ci-dessous, vous devrez répondre à la question :

\begin{objectif}
Quelle est la taille d'une molécule d'huile ?
\end{objectif}

\section*{L'expérience historique de Benjamin Franklin}

Au $\text{XVIII}^\text{ème}$ siècle, Benjamin Franklin se promène au bord de l'étang de Clapham en Angleterre, et décide de verser un peu d'huile dans l'eau. Il observe alors qu'une tache se forme à la surface et s'étend rapidement jusqu'à couvrir presque un quart de la surface du plan d'eau.

Il faudra attendre 1890 pour que Lord John Rayleigh reprenne cette expérience et en déduise la taille des molécules d'huile.

\begin{multicols}{2}
\center
\includegraphics[height=150pt]{images/franklin_lake.png}

L'étang agité un jour venteux.

\includegraphics[height=150pt]{images/franklin_lake_oiled.png}

La tache d'huile forme une étendue lisse.
\end{multicols}

\begin{doc}
\textbf{: extrait d'une lettre de Benjamin Franklin à la Royal Society (1774)}
\begin{multicols}{2}
\noindent
Texte original :

\textit{ \og
At length at Clapham where there is, on the common, a large pond, which I observed to be one day very rough with the wind, I fetched out a cruet of oil, and dropped a little of it on the water.
I saw it spread itself with surprising swiftness upon the surface.
The oil, though not more than a teaspoonful, produced an instant calm over a space several yards square, which spread amazingly and extended itself gradually until it reached the leeside, making all that quarter of the pond, perhaps half an acre, as smooth as a looking glass.
\fg{} }

\noindent
Traduction :

Enfin à Clapham où il y a, sur la commune, un grand étang que j'observai agité un jour de grand vent, je cherchai une burette d'huile et en laissai tomber un peu sur l'eau.
Je la vis se répandre sur la surface avec une rapidité surprenante.
L'huile, bien que moins d'une cuillère à café, produisit un calme immédiat sur une surface de plusieurs mètres carrés, qui se propagea incroyablement et s'étendit progressivement jusqu'à la côte rendant ce quart de l'étang, peut-être 2000\,m\textsuperscript{2}, aussi lisse qu'un miroir.
\end{multicols}
\end{doc}

\section*{Aide}

Vous vous mettez dans la peau de Lord John Rayleigh et exploitez l'expérience de Benjamin Franklin pour répondre à la question.
Comme tout scientifique qui se respecte, Lord Rayleigh répond à la question en respectant les principales étapes de la \textbf{démarche scientifique} :
\begin{enumerate}
\item \textbf{Hypothèse}.
Donnez votre hypothèse et justifiez-la : \og Je pense que ... car ... \fg{}.
\item \textbf{Protocole}.
Mettre en place un protocole pour valider (ou invalider !) votre hypothèse :
\begin{itemize}
\item[•] écrire en quelques lignes ce qu'a fait Benjamin Franklin ;
\item[•] établir une liste du matériel, comme si vous vouliez reproduire l'expérience ;
\item[•] réaliser l'expérience : cette fois, c'est Benjamin Franklin qui l'a faite ;
\item[•] indiquer les observations utiles : schéma et observations (à l'aide du schéma narratif par exemple) ;
\item[•] relever les mesures utiles, d'après les observations de Benjamin Franklin ;
\item[•] faire les calculs nécessaires.
\end{itemize}
\item \textbf{Conclusion}. Pour terminer le compte-rendu :
\begin{itemize}
\item[•] donner les conclusions en reprenant ce qui a été trouvé dans le protocole ;
\item[•] dire si les conclusions sont en accord avec votre hypothèse ;
\item[•] répondre à la question posée !
\end{itemize}
\end{enumerate}

\section*{Donnée}

$ \unit{1}{mL} = \unit{1\times 10^{-6}}{m\cubed}$

\section*{Évaluation}

L'évaluation de votre travail se fera sur la base des compétences mobilisées pour répondre à la question posée.
Vous pouvez vérifiez que vous remplissez les différents critères en vous reportant à la grille ci-dessous. 

\begin{center}
\begin{tabular}{l|l}
\textbf{Compétences} & Aptitudes à vérifier \hfill \textbf{Suis-je capable de ... ?} \\
\hline
\hline
\app				 		& Me servir correctement des ressources disponibles (doc, énoncé, ...) \\
		         			& Choisir les informations qui me seront utiles \\
							& Faire un schéma de l'expérience \\
							& Évaluer quantitativement les grandeurs physiques inconnues et non précisées \\
\hline
\anarai		  		& Faire une hypothèse, la justifier \\
							& Justifier le protocole choisi \\
      						& Donner des conclusions à l'activité \\
\hline
\rea     				& Réaliser correctement les calculs analytiques et/ou numériques \\
\hline
\val		      			& Dire si mes résultats sont en accord avec ceux attendus \\
			       			& Avoir un regard critique sur mes résultats \\
\hline
\com				  	& Rendre compte de façon écrite ou orale \\
		         			& Utiliser un vocabulaire et des modes de représentation adaptés \\
\hline
\rco         			& Restituer ses connaissances \\
\end{tabular}
\end{center}



\end{document}