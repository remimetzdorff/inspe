\documentclass[12pt,a4paper]{article}
\usepackage[utf8]{inputenc}
\usepackage[french]{babel}
\usepackage[T1]{fontenc}
\usepackage{amsmath}
\usepackage{amsfonts}
\usepackage{amssymb}
\usepackage{graphicx}
\usepackage[left=2cm,right=2cm,top=2cm,bottom=2cm]{geometry}

\author{Jean Fatet}
\title{Didactique et histoire des sciences}
\date{}

\begin{document}

\maketitle

\emph{04/09/2020}

\paragraph{Objectif :}
Construire une séance/une séquence pédagogique basée sur l'histoire des sciences et l'épistémologie.
Idéalement, il faudrait le faire au premier trimestre pour pouvoir faire un retour.
Pour le deuxième semestre, on choisit un instrument ancien ou une famille d'instrument, les remettre en état et faire des expériences pour comprendre comment ils étaient utilisés dans la recherche et l'enseignement.

\paragraph{Évaluation :}
Sous forme d'un dossier autour de cette séquence avec un bilan du déroulé pour le premier semestre, puis sur autour des instruments choisis au deuxième semestre.
Dans le rapport : présentation du programme en justifiant la pertinence d'une approche historique, recherche secondaire avec les textes les auteurs principaux pour reconstruire l'histoire du sujet.
On veut identifier une source primaire, un texte original et travailler à partir de cette source.
Présenter l'auteur, le contexte.
Construire et préparer une séance ou une séquence (pas forcément longue) utile pour le stage en lycée.
Réaliser l'analyse a priori : projet réel du déroulé du cours dans lequel on scénarise la séance avec les temps, les difficultés possibles) : il faut terminer une activité pour que les élèves sache quoi faire si on leur demande \og Qu'as tu fais aujourd'hui ? \fg{}.
Réaliser l'analyse a posteriori : comparer ce qu'on avait prévu à ce qui s'est réellement passé.
Idéalement, ce document pourrait servir à d'autres enseignants qui souhaitent réaliser la même séquence, il pourrait être publié dans une recueil pour réaliser un livre à destination des profs.
\end{document}