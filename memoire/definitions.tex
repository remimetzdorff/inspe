\documentclass[12pt,a4paper]{article}
\usepackage[utf8]{inputenc}
\usepackage[french]{babel}
\usepackage[T1]{fontenc}
\usepackage{amsmath}
\usepackage{amsfonts}
\usepackage{amssymb}
\usepackage{graphicx}
\usepackage[left=2cm,right=2cm,top=2cm,bottom=2cm]{geometry}

\begin{document}

\paragraph{Compétence :}
Selon le bulletin officiel n° 17 du 23 avril 2015, \og une compétence est l'aptitude à mobiliser ses ressources (connaissances, capacités, attitudes) pour accomplir une tâche ou faire face à une situation complexe ou inédite. \fg{} 

\paragraph{Tâche complexe :}
C'est une tâche mobilisant des ressources internes (connaissances, culture, capacités, attitudes, vécu, etc.) et externes (aides méthodologiques,  protocoles, fiches techniques, ressources documentaires, etc.) qui permet l'expression d'une compétence.
L'objectif est clairement identifié mais le chemin de résolution n'est pas donné.

\paragraph{Situation-problème :}
Selon Courtillot, D. et al. (2004), une situation problème a pour objectif d'invalider une conception des élèves.
Pour l'enseignant, elle nécessite d'identifier clairement la conception attaquée de manière à choisir une situation qui la fasse émerger chez l'élève.
Cette situation amène un questionnement qui invite l'élève à émettre des hypothèses.
Ces hypothèses \og doivent nécessairement être invalidées dans la phase de test pour les mettre à l'épreuve et dépasser la conception visée. \fg{}

\paragraph{Résolution de problème :}
Ce type d'activité est inscrit dans le préambule du programme de terminale S.
Il en est question dans le rapport du GRIESP intitulé Résoudre un problème de physique-chimie dès la seconde publié en juillet 2014 : \og La résolution de problèmes contribue à la formation des élèves aux compétences de la démarche scientifique. \fg{}

\paragraph{Problème ouvert :}
Comme dans Boilevin, J.-M. (2005), on peut le définir par opposition au problème fermé plus classique.
L'énoncé ne contient pas de données, et ne présuppose l'utilisation d'aucun modèle.
Le choix du modèle par l'élève peut conduire à différentes résolutions.

\end{document}